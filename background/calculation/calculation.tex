%----------------------------------------------------------------------------------------
%	PACKAGES AND OTHER DOCUMENT CONFIGURATIONS
%----------------------------------------------------------------------------------------


\documentclass[twoside]{article}
 
\usepackage{textcomp}


\usepackage{lipsum} % Package to generate dummy text throughout this template

\usepackage[sc]{mathpazo} % Use the Palatino font
\usepackage[T1]{fontenc} % Use 8-bit encoding that has 256 glyphs
\linespread{1.05} % Line spacing - Palatino needs more space between lines
\usepackage{microtype} % Slightly tweak font spacing for aesthetics

\usepackage{amsmath}

\usepackage[hmarginratio=1:1,top=32mm,columnsep=20pt]{geometry} % Document margins
\usepackage{multicol} % Used for the two-column layout of the document
\usepackage[hang, small,labelfont=bf,up,textfont=it,up]{caption} % Custom captions under/above floats in tables or figures
\usepackage{booktabs} % Horizontal rules in tables
\usepackage{float} % Required for tables and figures in the multi-column environment - they need to be placed in specific locations with the [H] (e.g. \begin{table}[H])
\usepackage{hyperref} % For hyperlinks in the PDF

\usepackage{lettrine} % The lettrine is the first enlarged letter at the beginning of the text
\usepackage{paralist} % Used for the compactitem environment which makes bullet points with less space between them

\usepackage{abstract} % Allows abstract customization
\renewcommand{\abstractnamefont}{\normalfont\bfseries} % Set the "Abstract" text to bold
\renewcommand{\abstracttextfont}{\normalfont\small\itshape} % Set the abstract itself to small italic text

\usepackage{titlesec} % Allows customization of titles
%\renewcommand\thesection{\Roman{section}} % Roman numerals for the sections
%\renewcommand\thesubsection{\Roman{subsection}} % Roman numerals for subsections
%\titleformat{\section}[block]{\large\scshape\centering}{\thesection.}{1em}{} % Change the look of the section titles
\titleformat{\subsection}[block]{\large}{\thesubsection.}{1em}{} % Change the look of the section titles

\usepackage{fancyhdr} % Headers and footers
\pagestyle{fancy} % All pages have headers and footers
\fancyhead{} % Blank out the default header
\fancyfoot{} % Blank out the default footer
%\fancyhead[C]{Convolution $\bullet$ August 2019} % Custom header text
\fancyfoot[RO,LE]{\thepage} % Custom footer text

\usepackage[english]{babel}
\usepackage{graphicx}


%change the margins
\addtolength{\oddsidemargin}{-.875in}
\addtolength{\evensidemargin}{-.875in}
\addtolength{\textwidth}{1.75in}



%----------------------------------------------------------------------------------------
%	TITLE SECTION
%----------------------------------------------------------------------------------------

<<<<<<< HEAD
\title{\vspace{-15mm}\fontsize{24pt}{10pt}\selectfont\textbf{Current Approach to Muon Convolution}} % Article title
=======
\title{\vspace{-15mm}\fontsize{24pt}{10pt}\selectfont\textbf{An Approach to Muon Convolution}} % Article title
>>>>>>> temp

\author{
\large
%\textsc{Jason Bono}\thanks{Thanks to David Brown of Berkley Lab}\\[2mm] % Your name
\textsc{Jason Bono}\\[2mm] % Your name
%\normalsize Fermilab \\ % Your institution
%\normalsize \href{mailto:jbono@fnal.gov}{jbono@fnal.gov} % Your email address
\vspace{-5mm}
}
\date{}

%----------------------------------------------------------------------------------------

\begin{document}

\maketitle % Insert title

\thispagestyle{fancy} % All pages have headers and footers

%----------------------------------------------------------------------------------------
%	ABSTRACT
%----------------------------------------------------------------------------------------

%\begin{abstract}
%
%\noindent The accurate reconstruction of a particle track using drift chambers requires knowledge of the paths taken by electrons within a gas under the influence of an electric and magnetic field. At present, the magnetic field, which as we will show increases the total drift time by up to 12\% for Mu2e, is not incorporated in the tracking algorithm. In this document, we obtain the classical equations of motion for the steady state in the macroscopic limit, introduce a method for the development of approximation schemes relevant to most straw tube trackers, and derive the three dimensional drift paths, thereby deriving correction factors to be applied to the simulation and reconstruction algorithms for Mu2e. 
%
%\end{abstract}

%----------------------------------------------------------------------------------------
%	ARTICLE CONTENTS
%----------------------------------------------------------------------------------------


<<<<<<< HEAD
\section{Frameworks}
=======
\section{Framework}
>>>>>>> temp
\begin{multicols}{2} % Two-column layout throughout the main article text
\subsection{Moments Method}
\subsubsection{Average Field}

We want the average scalar (vertical component) B field experienced by the muons
\begin{equation}
\left\langle B \right\rangle = B(\textbf{x},t) \otimes M(\textbf{x},t)
\end{equation}
This involves convolving the spacial and time structure of the muon beam and the B field. The spacial structures can be expanded into moments as $B(\textbf{x},t) = \Sigma_n b_n(t)$ and $M(\textbf{x},t) = \Sigma_n m_n(t)$ where $b_n(t)$  and $m_n(t)$ are the nth field and muon moment. However,  there will be different uncertainty on the measurement of each moment and may want to weight the n contributions to final result accordingly, i.e. take the weighted sum
\begin{equation}
B(\textbf{x},t) = \Sigma_n b_n(t) w^{\text{b}}_n 
\end{equation}
and 
\begin{equation}
M(\textbf{x},t) = \Sigma_n m_n(t) w^{\text{m}}_n
\end{equation}
where $w^{\text{x}}_n$ implicitly obeys some normalization condition, such as 
\begin{equation}
\Sigma^N_{n=1}  w^{\text{x}}_n = N
\end{equation}
Also note that $w^x_n$ has no time structure. 
The average experienced B field can now be expressed as the weighted sum of products of the moments
\begin{equation}
\left\langle B \right\rangle = 
\Sigma_n b_n(t) w^{\text{b}_n}  \otimes  
\Sigma_n m_n(t) w^{\text{m}_n} = 
%\Sigma_n (b_{n}(t) \otimes m_{n}(t))
\end{equation}
or
\begin{equation}
\left\langle B \right\rangle = 
\Sigma_n w^{\text{b}}_n w^{\text{m}}_n (b_{n}(t) \otimes m_{n}(t))
\end{equation}
or, letting $w_n = w^{\text{b}}_n w^{\text{m}}_n$,
\begin{equation}
\left\langle B \right\rangle =  \Sigma_n w_n (b_{n}(t) \otimes m_{n}(t))
\end{equation}
We can express what is left of the convolution as a weighted average
\begin{equation}
b_{n}(t) \otimes m_{n}(t) = 
\frac{ \int b_{n}(t) m_{n}(t) c_n(t) dt}{\int c_n(t) dt}
\end{equation}
where $c_n(t)$ is, for now, a general weighting factor that can absorb various effects, such as the changing number of muons, field uncertainties, etc. Identifying $c_n(t)$ exactly is reserved for a later section. Substituting the above expression in for $\left\langle B \right\rangle$, we get
\begin{equation}
\left\langle B \right\rangle =
 \Sigma_n w_n \frac{ \int b_{n}(t) m_{n}(t) c_n(t) dt}{\int c_n(t) dt}
\end{equation}
<<<<<<< HEAD


=======
>>>>>>> temp
Provided time is discretized finely enough so that, within any single interval, the spacial structures are effectively constant, we can write
\begin{equation}
	\left\langle B \right\rangle = 
	\Sigma_{n} w_n \frac{\Sigma_{t} b_{nt} m_{nt} c_{nt}}{\Sigma_t c_{nt}}
\end{equation}
<<<<<<< HEAD

=======
>>>>>>> temp
Letting $\Sigma_{t} c_{nt} = C_n $ gives
\begin{equation}
\boxed{
\left\langle B \right\rangle =
\Sigma_{n}[ \frac{w_n}{C_n} (\Sigma_{t} b_{nt} m_{nt} c_{nt})].
}
\end{equation}
Note that  we could let $C_n$ absorb $w_n$, but to avoid confusion, we will not do so. Finally, we can write the contribution from a single (nth) moment as
\begin{equation}
\boxed{
	\left\langle B \right\rangle_n =
	[ \frac{w_n}{C_n} (\Sigma_{t} b_{nt} m_{nt} c_{nt})]
}
\end{equation}
and that contribution in a single time-slice at $\tau$ as 
\begin{equation}
\boxed{
		\left\langle B \right\rangle_{n\tau} =
		 \frac{w_n b_{n\tau} m_{n\tau} c_{n\tau}}{C_n}
}
\end{equation}


\end{multicols}
\newpage
%%%%%%%%%%%%%%%%%%%%%
%%%%%%%%%%%%%%%%%%%%%
\subsubsection{Error in Average Field}
Assume that the muon and field structures ($m_{nt}$ and $b_{nt}$) are uncorrelated, and that the uncertainties on $C_n$ and $w_n$ are zero. The variance in the nth moment at time $t$, $\sigma^2_{nt} \equiv (\Delta \left\langle B \right\rangle_{nt})^2$, is 
<<<<<<< HEAD

=======
>>>>>>> temp
\begin{equation}
\sigma^2_{nt} =
(\frac{d \left\langle B \right\rangle_{nt}}{db_{nt}}\Delta b_{nt})^2 +
(\frac{d \left\langle B \right\rangle_{nt}}{dm_{nt}}\Delta m_{nt})^2 + 
(\frac{d \left\langle B \right\rangle_{nt}}{dc_{nt}}\Delta c_{nt})^2 
\end{equation}
<<<<<<< HEAD

=======
>>>>>>> temp
or 
\begin{equation}
\boxed{
\sigma^2_{nt} =
\frac{w^2_n}{C^2_n}[(m_{nt} c_{nt} \Delta b_{nt})^2 + (b_{nt} c_{nt} \Delta m_{nt})^2  + (b_{nt} m_{nt} \Delta c_{nt})^2 ]
}
\end{equation}
<<<<<<< HEAD

substituting the above expression into
=======
Substituting the above expression into
>>>>>>> temp
\begin{equation}
\sigma^2_{n} = \Sigma_t  \sigma^2_{nt} 
\end{equation}
gives 
\begin{equation}
\boxed{
\sigma^2_{n} = \frac{w^2_n}{C^2_n} \Sigma_t  [(m_{nt} c_{nt} \Delta b_{nt})^2 + (b_{nt} c_{nt} \Delta m_{nt})^2  + (b_{nt} m_{nt} \Delta c_{nt})^2 ]
}
\end{equation}
Similarly, substituting the above expression into
\begin{equation}
\sigma^2 = \Sigma_n  \sigma^2_{n} 
\end{equation}
gives
\begin{equation}
\boxed{
	\sigma^2 = \Sigma_n [\frac{w^2_n}{C^2_n} \Sigma_t  [(m_{nt} c_{nt} \Delta b_{nt})^2 + (b_{nt} c_{nt} \Delta m_{nt})^2  + (b_{nt} m_{nt} \Delta c_{nt})^2 ]]
}
\end{equation}


<<<<<<< HEAD
=======
\subsubsection{Moments Method in Practice}


>>>>>>> temp


\subsection{The Matrix Method}
\subsubsection{Matrix Average Field}
As an alternative method, one can write 
\begin{equation}
M(\textbf{x},t) = \frac{P(\textbf{x},t)n(t)}{\int n(t) dt} 
\end{equation}
where $P$ is a 2d projected ($\textbf{x}=r,y$) normalized muon distribution, and n(t) represents the number of muons in that projection at time $t$. Calling $N$ the integral over $n(t)$,  Equation 1 becomes
\begin{equation}
\left\langle B \right\rangle = \frac{B(\textbf{x},t) \otimes P(\textbf{x},t)n(t)}{N}
\end{equation}
Which can also be expressed as 
\begin{equation}
\left\langle B \right\rangle = \frac{\int \int \int  B(r,y,t) P(r,y,t)n(t)dtdrdy  }  {N}
\end{equation}
Discretizing $r$,$y$, and $t$ into $R$,$Y$, and $T$ segments gives
\begin{equation}
\label{dis}
\boxed{
\left\langle B \right\rangle = \frac{\Sigma_t \Sigma_r \Sigma_y    B_{ryt} P_{ryt}n_t }  {N}
}
\end{equation}

\paragraph{Interpretation:}
Equation \ref{dis}  is all that's needed, but for interpretation, we can view $B_{ryt} P_{ryt}$ as the {r,y,t} element of Hadamard product of the two matrices $B$ and $P$, which have $R$ rows, $Y$ columns, and $T$ layers. Let $H$ be that $R \times Y \times T$  Hadamard product
\begin{equation}
H = B \odot P
\end{equation}
then
\begin{equation}
\left\langle B \right\rangle = \frac{  \Sigma_r \Sigma_y \Sigma_t  n_t  H_{ryt} }  {N}
\end{equation}
where $\Sigma_t  n_t  H_{ryt}/N$ can been seen as the $n$ weighted average of all layers of $H$. Call that weighted average $\bar{H} $,
\begin{equation}
\bar{H}_{ry} \equiv \frac{ \Sigma_t  n_t  H_{ryt} }  {N}
\end{equation}
So we finally have that the convoluted B field is just the sum of elements of $\bar{H}$, or
\begin{equation}
\boxed{
\left\langle B \right\rangle =  \Sigma_r \Sigma_y \bar{H}_{ry}
}
\end{equation}

\paragraph{Special Case:}
Note that, in the special case where $P_{ryt}$ has no time dependence, then we can rewrite Equation \ref{dis} as 
\begin{equation}
\left\langle B \right\rangle =  \Sigma_r \Sigma_y P_{ry} \frac{\Sigma_t B_{ryt} n_t }  {N}
\end{equation}
where the fraction can be thought of as the $n$ weighted average of the B field. Explicitly  letting
\begin{equation}
\hat{B} \equiv \frac{\Sigma_t B_{ryt} n_t }  {N}
\end{equation}
gives
\begin{equation}
\boxed{
\left\langle B \right\rangle =  \Sigma_r \Sigma_y P_{ry} \hat{B}_{ry}
}
\end{equation}
ie the sum of all elements of the Hadamard product between the normalized muon distribution and the $n$ weighted average B field.


\subsubsection{Matrix Average Field Error}
Assuming independence, the total variance in Equation \label{dis} can be written as
\begin{equation}
	\sigma^2 = 
	\Sigma_r \Sigma_y \Sigma_t \sigma^2_{ryt}
\end{equation}
where 
\begin{equation}
 \sigma^2_{ryt} = \frac{1}{N^2} [( P_{ryt} n_t \Delta B_{ryt})^2  + (B_{ryt} n_t\Delta P_{ryt})^2 + (B_{ryt} P_{ryt}\Delta n_{t})^2]
\end{equation}
or 
\begin{equation}
\boxed{
\sigma^2 = 
\frac{1}{N^2} \Sigma_r \Sigma_y \Sigma_t  [( P_{ryt} n_t \Delta B_{ryt})^2  + (B_{ryt} n_t\Delta P_{ryt})^2 + (B_{ryt} P_{ryt}\Delta n_{t})^2]
}
\end{equation}

Once again, the uncertainty on N is zero by definition since it's a normalization factor for the sum over $n$. $\Delta B_{ryt}$ comes from uncertainty in the field determination. $\Delta P_{ryt}$ comes from uncertainty in the tracker-determined beam distribution.  $\Delta n_{t}$ comes from uncertainty in the CTAGs in each time slice. Note that uncertainties arising from spacial discretization are correlated from cell to cell and would therefore need to be treated in detail. However, this can be circumvented by fine-graining. 



\paragraph{A Special Case:}
In the case where the uncertainty on $n$ and $P$ are zero, and that $P$ has no time dependence, we get
\begin{equation}
	\sigma^2 = 
	\Sigma_r \Sigma_y P^2_{ry} \frac{\Sigma_t  ( n^2_t \Delta B^2_{ryt})}{{N^2} }
\end{equation}
where the faction is the $n^2$ weighted mean of $\Delta B^2$

\subsubsection{Matrix Method in Practice}
Information on $P$ is provided by the tracker team as a weighted 2d histogram encoded into a text file. So we effectively get $P_{ryt} = P_{ry}$ directly. The tracker team has not assigned explicit bin-by-bin uncertainty so it's taken to be zero. The uncertainty in the muon distribution is implicit in its diffusion.

Information on $n_t$ is provided by the gm2\_online\_prod database. We assume that the uncertainty associated with $n_t$ is small enough to ignore. Any scale uncertainty cancels, and any uncertainty caused by discretization of time is highly correlated and would have a small overall effect.

Information on $B_{ryt}$ and $\Delta B_{ryt}$ are provided by the field team, and encoded as multipole strengths and uncertainties in distinct time layers.  We can use an analytical function to map the multipoles onto the {r,y} space in each time layer. We can map the uncertainties in the multipoles in the same way. We can further map the two space-continuous variables onto a grid giving$B_{ryt}$ and  $\Delta B_{ryt}$. 

The formula to convert the multipole moments, $b_i$, to the magnetic field, $B(r,\theta)$, is taken as definition since it is what has been used by the field team to describe the field.

\begin{equation}
B(r,\theta) =  b_1 +  \sum_{n=1} R^{n} (b_{2n} \cos(n \theta)  + b_{2n+1} \sin(n \theta )   )
\end{equation}
where $b_n$ has units of ppm. Also, note that $R$ is dimensionless, with 
\begin{equation}
R = \frac{r}{4.5 \text{ cm}}  = \frac{\sqrt{x^2 + y^2}}{4.5 \text{ cm}} 
\end{equation}
Where $r$,$x$, and$y$ are in cm. Subbing in 
\begin{equation}
\theta = \arctan \frac{y}{x}
\end{equation}
gives
\begin{equation}
B(x,y)  =  b_1 +  \sum_{n=1} (\frac{\sqrt{x^2 + y^2}}{4.5})^{n} (b_{2n} \cos(n \arctan \frac{y}{x})  + b_{2n+1} \sin(n  \arctan \frac{y}{x} )   )
\end{equation}



%%%%%%%%%%%%%%%%%%%%%%%%
%%%%%%%%%%%%%%%%%%%%%%%%%%%%
%%%%%%%%%%%%%%%%%%%%%%%%%%%%
\newpage
\section{Application}
\subsection{A List}
%\begin{multicols}{2}
Applying the framework is just a matter of connecting the familiar observables to the variables introduced in this document. Below is a list of such connections. The list is only intended to be instructive for now, so many items are left out. Eventually, we will want to define everything exactly.
\begin{itemize}
	\item $b_{1t}$  The dipole field in time-slice $t$
	\item $m_{1t}$ The muon beam's dipole moment in time-slice $t$ (always equal to unity)
	\item $b_{2t}$ The normal-quadrupole  field in time-slice $t$
	\item $m_{2t}$ The muon beam's mean horizontal position in time-slice $t$
	 \item $\Delta b_{2t}$ The error in the normal-quadrupole  field in time-slice $t$
	 \item $\Delta b_{2t}$ The spread (RMS) in the muon beam's mean horizontal position in time-slice $t$	
	 \item $c_{nt}$ Nominally identified with CTAGs in time-slice $t$
	 \begin{itemize}
	 	 \item Then $c_{nt}$  is reduced to $c_{t}$ and $C_{n}$ to $C$, where $C$ is the total number of CTAGs integrated over the time interval in question 
	 	 \item However, if we want to do an uncertainty weighted average, $c_{nt}$ is the natural variable to do this with
	 	 \item  $\Delta c_{nt}$ Will probably be set to zero, but this needs to be thought about more
	 \end{itemize}
	 \item  $w_{n}$ Nominally set to 1 or 0 for moments that are considered or not considered in the final sum over n
	 \begin{itemize}
	 	 \item However, we may want to explicitly weight some moments more than others, and $w_{n}$ is the natural variable to do this with
	 	 \item  $\Delta w_{n}$ has been assumed to be zero and will almost certainly remain this way
	 \end{itemize}
	
\end{itemize}
%\end{multicols}

\newpage
\subsection{Simple Examples}
%\begin{multicols}{2}
\subsubsection{Dipole}
Here we calculate  $\left\langle B \right\rangle_1$ and $\sigma_1$. We have
\begin{itemize}
	\item $b_{1t} = \text{D}$ (dipole)
	\item $\Delta b_{1t} = \text{eD}$ 
	\item $c_{1t} = c_t = \text{ctag}$ (number of muons)
	\item $\Delta c_{1t} = \Delta c_t = 0$ 
	\item $C_{1} = C = \text{ctag\_total}$  (cumulative sum of number of muons)
	\item $m_{1t} = 1$  (by definition)
	\item $\Delta m_{1t} = 0$.
	\item $w_{1} = 1$ (by definition)
\end{itemize}
Subbing some of these values into the relevant equations, we get
\begin{equation}
		\left\langle B \right\rangle_1 =
		 \frac{\Sigma_{t} b_{1t} c_{t}}{C} = 
		  \frac{\Sigma_{t} D_t \text{ctag}_{t}}{\text{ctag\_total}}
\end{equation}
and 
\begin{equation}
	\sigma_{1} = \frac{1}{C}\sqrt{\Sigma_t  [(c_{t} \Delta b_{1t})^2 ]} = 
	\frac{\sqrt{\Sigma_t  [(\text{eD}_t \text{ctag}_t)^2 ]} }{\text{ctag\_total}}
\end{equation}





\subsubsection{Normal Quadrupole}
Here we calculate  $\left\langle B \right\rangle_2$ and $\sigma_2$. We have
\begin{itemize}
	\item $b_{2t} = \text{NQ}$ (normal quadrupole))
	\item $\Delta b_{2t} = \text{eNQ}$ 
	\item $c_{2t} = c_t = \text{ctag}$ (number of muons)
	\item $\Delta c_{2t} = \Delta c_t = 0$ 
	\item $C_{2} = C = \text{ctag\_total}$  (cumulative sum of number of muons)
	\item $m_{2t}$  is the mean horizontal beam position
	\item $\Delta m_{2t}$ is the horizontal RMS of the beam about its mean 
	\item $w_{2} = 1$ (by definition)
\end{itemize}
We get
\begin{equation}
\left\langle B \right\rangle_2 =
\frac{\Sigma_{t} m_{2t} b_{2t} c_{t}}{C} = 
\frac{\Sigma_{t} m_{2t} NQ_t \text{ctag}_{t}}{\text{ctag\_total}}
\end{equation}

\begin{equation}
	\sigma_{2} =
	 \frac{1}{C} \sqrt{\Sigma_t  [(m_{2t} c_{t} \Delta b_{2t})^2 + (b_{2t} c_{t} \Delta m_{2t})^2  ]} =
	 \frac{\sqrt{\Sigma_t  [(\text{eNQ}_t  \text{ctag}_t m_{2t} )^2 + (\text{NQ}_t  \text{ctag}_t \Delta m_{2t})^2  ]} }{\text{ctag\_total}}
\end{equation}



\subsubsection{Skew Quadrupole}
Here we calculate  $\left\langle B \right\rangle_3$ and $\sigma_3$. We have
\begin{itemize}
	\item $b_{3t} = \text{SQ}$ (skew quadrupole)
	\item $\Delta b_{3t} = \text{eSQ}$ 
	\item $c_{3t} = c_t = \text{ctag}$ (number of muons)
	\item $\Delta c_{3t} = \Delta c_t = 0$ 
	\item $C_{3} = C = \text{ctag\_total}$  (cumulative sum of number of muons) 
	\item $m_{3t}$  is the mean vertical beam position
	\item $\Delta m_{3t}$ is the vertical RMS of the beam about its mean 
	\item $w_{3} = 1$ (by definition)
\end{itemize}
We get
\begin{equation}
\left\langle B \right\rangle_3 =
\frac{\Sigma_{t} m_{3t} b_{3t} c_{t}}{C} = 
\frac{\Sigma_{t} m_{3t} SQ_t \text{ctag}_{t}}{\text{ctag\_total}}
\end{equation}

\begin{equation}
\sigma_{3} =
\frac{1}{C} \sqrt{\Sigma_t  [(m_{3t} c_{t} \Delta b_{3t})^2 + (b_{3t} c_{t} \Delta m_{3t})^2  ]} =
\frac{\sqrt{\Sigma_t  [(\text{eSQ}_t  \text{ctag}_t m_{3t} )^2 + (\text{SQ}_t  \text{ctag}_t \Delta m_{3t})^2  ]} }{\text{ctag\_total}}
\end{equation}


\subsubsection{Adding contributions from the first three moments}
In the above examples we found the contributions from moments 1,2, and 3. To get the total contribution, we take the simple sum
\begin{equation}
\left\langle B \right\rangle =
\Sigma^3_{n=1} \left\langle B \right\rangle_n =
\frac{\Sigma_{t} c_t [(b_{1t}) + (m_{2t} b_{2t}) + (m_{3t} b_{3t})] }{C} 
\end{equation}
For the total variance, we also take the simple sum,
\begin{equation}
\sigma^2 = 
\Sigma^3_{n=1} \sigma^2_n = 
\frac{1}{C^2} \Sigma_t c^2_t [(\Delta b_{1t})^2 +
(m_{2t} \Delta b_{2t})^2 + (b_{2t} \Delta m_{2t})^2 + 
(m_{3t} \Delta b_{3t})^2 + (b_{3t} \Delta m_{3t})^2 
] 
\end{equation}
and so the error is
\begin{equation}
\sigma = 
\frac{\sqrt{\Sigma_t c^2_t [(\Delta b_{1t})^2 +
	(m_{2t} \Delta b_{2t})^2 + (b_{2t} \Delta m_{2t})^2 + 
	(m_{3t} \Delta b_{3t})^2 + (b_{3t} \Delta m_{3t})^2 
	]}}{C}
\end{equation}


Note that, in general, the values $w_{1,2,3}$ could have been made unequal in in order to weight the contributions from some moments over others, possibly based on sources of uncertainty. Similarly, identifying $c_{nt}$ with ctags alone allowed for its contraction into $c_t$. This may not always be the case. For example, we may allow the $c_{nt}$ weighting factor to incorporate uncertainty from the field determination, which may have distinct effects on the various moments.




\newpage
\subsection{Sanity Checks }
\subsubsection{Invariance  to Total CTAGs}
The quantity $c_{nt}$ will always incorporate the number of muons (though CTAGS), even if it incorporates other factors simultaneously. Intuitively, the average field experience by the muons should be invariant to the total number of muons. We can represent an error factor and offset, $f$, in our ability to count muons. We can then represent this as
\begin{equation}
c_{nt} \to fc_{nt}
\end{equation}
which implies 
\begin{equation}
C_{n} \to fC_{n}
\end{equation}
which implies
\begin{equation}
	\left\langle B \right\rangle =
	\Sigma_{n}[ \frac{w_n}{C_n} (\Sigma_{t} b_{nt} m_{nt} c_{nt})] \to
	\Sigma_{n}[ \frac{w_n}{f C_n} (\Sigma_{t} b_{nt} m_{nt} f c_{nt})] = 
	\left\langle B \right\rangle
\end{equation}
ie the calculated field is unchanged.


Furthermore, for the uncertainty, we get $\Delta c_{nt} \to f\Delta c{nt}$, which implies
\begin{equation}
\begin{split}
	\sigma^2 = \Sigma_n [\frac{w^2_n}{C^2_n} \Sigma_t  [(m_{nt} c_{nt} \Delta b_{nt})^2 + (b_{nt} c_{nt} \Delta m_{nt})^2  + (b_{nt} m_{nt} \Delta c_{nt})^2 ]] \to \\ 
	\Sigma_n [\frac{w^2_n}{f^2 C^2_n} \Sigma_t  [(m_{nt} f c_{nt} \Delta b_{nt})^2 + (b_{nt} f c_{nt} \Delta m_{nt})^2  + (b_{nt} m_{nt} f \Delta c_{nt})^2 ]] = 
	\sigma^2 
\end{split}
\end{equation}
ie the calculated uncertainty in the field is unchanged.


\subsubsection{Invariance to Sample Rate}
Analytically, our calculations should also not be affected by the sample rate, provided it is fine enough. Note that $b_{nt}$ always represents the time averaged magnetic field in time-slice $t$, so its value does not change when re sampled. $m_{nt}$ always represents a geometric representation of the beam, and its value also does not change when re-sampled. $c_{nt}$ on the other hand always incorporates the (sometimes fractional) number of muons in time-slice $t$. It is therefore a linearly cumulative variable (its value changes proportionally to the sample rate). $c_{nt}$ may also incorporate other variables such as the error in $b_{nt}$ and $m_{nt}$, but these are not cumulative. Provided these statements hold true, the product $b_{nt} m_{nt}  c_{nt}$ is linearly cumulative variable in $t$. Therefore
\begin{equation}
	\left\langle B \right\rangle_{nt} =
	\frac{w_n b_{nt} m_{nt} c_{nt}}{C_n}
\end{equation}
is also a linearly cumulative variable and thus is invariant to sampling rate, provided the sampling is done over a fixed time interval.

By similar reasoning, $\Delta b_{nt}$ and $\Delta m_{nt}$ are non-cumulative variables, so if $\Delta c_{nt}$ is linearly cumulative, zero, or negligible, then
\begin{equation}
	\sigma_{nt} =
	\frac{w_n}{C_n}\sqrt{[(m_{nt} c_{nt} \Delta b_{nt})^2 + (b_{nt} c_{nt} \Delta m_{nt})^2  + (b_{nt} m_{nt} \Delta c_{nt})^2 ]}
\end{equation}
is also linearly cumulative and therefore invariant. In most cases, though perhaps not all, the restriction on $\Delta c_{nt}$ will hold.



\newpage
\subsection{ Questions }
Below are some questions that need to be answered:

\begin{itemize}
	\item What are the problems with the approach to average and error on average?
	\item Are non-binary weights for $w_n$ needed?
	\begin{itemize}
		\item If so, what normalization condition should we use? 
	\end{itemize}
	\item Should anything other than ctags be put into $c_{tn}$? Eg the field uncertainty?
	\item Does the accumulation of error over time, as it's formulated make sense?
		\begin{itemize}
		\item Does our error calculation have built-in invariance to sample-rate? (see sanity checks)
		\item Does our field and error calculation have built-in invariance to total CTAGs? (see sanity checks)
		
	\end{itemize}
	\item How should we use the muon distribution to get the higher order moments (ie above quadrupole)
	
\end{itemize}





\end{document}






%
%
%
%\newpage
%\section{old stuff}
%We can put an upper bound on the error associated with the assumption of a constant $r$-component of velocity.  Consider the case of maximal deviation, \emph{i.e.} when the two fields are perpendicular. 
%The exact d-to-t function for the special case of perpendicular fields is begotten through the limit,
%\begin{equation}
%t_{\perp} = \lim_{\cos\phi\to 0} \frac{(\cos^2 \phi - 1) \arctan(d C \cos \phi ) + d C \cos \phi }{v_d C \cos^3 \phi}
%\end{equation}
%which converges to,
%\begin{equation}
%t_{\perp} = \frac{d}{v_d}( 1 + \frac{C^2 d^2}{3})
%\end{equation}
%Meanwhile, the approximate solution, namely that with constant $u_x$, is found directly from Equation~\ref{eq:components},
%\begin{equation}
%t^\prime_{\perp} = \frac{d}{u_x} = \frac{d (1 + C^2 d^2)}{v_d}
%\end{equation}
%We can quantify the relative effect of integrating the reciprocal of velocity with respect to $r$ by taking the ratio,
%\begin{equation}
%\frac{t_\perp}{t^\prime_{\perp}}  = \frac{1 + \frac{C^2 d^2}{3}}{1 + C^2 d^2}
%\end{equation}
%which is over a 10\% effect at $\zeta=0.5$, as can be seen from Figure~\ref{fig:ratio}.
%\begin{figure}[H]
%\label{fig:ratio}
%    \includegraphics[width=0.47\textwidth]{pics/ratio}
%    \caption{Here we have a worst case scenario for  the ratio of computed times vs $\zeta$. In mu2e,  0<$\zeta$<0.5}
%\end{figure}
%We we can therefore take the non-integrated d-to-t functions to good approximation. The expression for the d-to-t function is,
%\begin{equation}
%t(d,\phi) = \frac{d}{u_x} = \frac{d}{v_d}\bigg(\frac{1 + C^2 d^2}{1 + C^2 d^2 \cos^2 \phi}\bigg)
%\end{equation}
%The ratio of time calculated with to without the maximal lorentz effect is,
%\begin{equation}
%R_{\text{calc}} = \frac{1 + C^2 d^2}{1 + C^2 d^2 \cos^2 \phi}
%\end{equation}
%which is plotted in Figure~\ref{fig:calc} for $\phi = 0$
%\begin{figure}[H]
%\label{fig:calc}
%    \includegraphics[width=0.47\textwidth]{pics/calc}
%    \caption{The ratio of computed times with and without the Lorentz effect as a function of $\zeta$. In mu2e,  0<$\zeta$<0.5}
%\end{figure}
%The true effect should be the same expression obtained from integration. This is plotted in figure Figure~\ref{fig:true} for $\phi = 0$
%\begin{equation}
%R_{\text{true}} = 1 + \frac{C^2 d^2 \cos^2 \phi}{3}
%\end{equation}
%\begin{figure}[H]
%\label{fig:true}
%    \includegraphics[width=0.47\textwidth]{pics/true}
%    \caption{The ratio of computed times with and without the Lorentz effect as a function of $\zeta$. In mu2e,  0<$\zeta$<0.5}
%\end{figure}
%A reasonable middle ground is to compute the d-to-t function without integration, but using the time averaged value of $\zeta$ which is essentially the time averaged value of $r$ over the trajectory,
%\begin{equation}
%r_{\text{avg}} = \frac{\int r dt}{\int dt} = \frac{\int \frac{r dr}{u_r}}{\int \frac{dr}{u_r}}
%\end{equation}
%which involves the same nasty expressions we set out to avoid. We can approximate $r_{\text{avg}}$ by making the assumption that the velocity decreases linearly with $r$. Under this assumption we get,
%\begin{equation}
%r_{\text{avg}}  = \frac{d}{2}
%\end{equation}
%Substituting this into ??? we get
%\begin{equation}
%\boxed{
%t(d,\phi) = \frac{d}{v_d}\bigg(\frac{1 + (\frac{C d}{2})^2}{1 + (\frac{C d  \cos \phi}{2})^2}\bigg)
%}
%\end{equation}
%which is plotted in Figure~????. In order to get the t-to-d function we must invert the above expression. We rewrite d-to-t function as, 
%\begin{equation}
%t(d_0,\phi) = \frac{d_0}{v_d}f(d_0)
%\end{equation}
%We want to find the function $g(t)$ such that,
%\begin{equation}
%d(t_0,\phi) = t_0v_d g(t_0)
%\end{equation}
%Substituting $t_0  = \frac{d_0}{v_d}$ gives,
%\begin{equation}
%d(t_0,\phi) = t_0v_d g(\frac{d_0}{v_d})
%\end{equation}
%In Equation ?? and ??, $f$ and $g$ can be thought of as contraction factors on the velocity. 
%\begin{equation}
%u_x = \frac{v_d}{f(d_0)} =   v_d g(\frac{d_0}{v_d})
%\end{equation}
%hence,
%\begin{equation}
%g(\frac{d_0}{v_d}) = f(d_0)
%\end{equation}
%With this transformation and the form of $f$, we can immediately write down the t-to-d function,
%\begin{equation}
%\boxed{
%d(t_0,\phi) = t_0 v_d \bigg(\frac{1 + (\frac{C t_0 v_d  \cos \phi}{ 2})^2}{1 + (\frac{C t_0 v_d}{2})^2}\bigg)
%}
%\end{equation}
%which is shown in Figure~\ref{fig:t2d}
%\begin{figure}[H]
%\label{fig:t2d}
%    \includegraphics[width=0.47\textwidth]{pics/t2d}
%    \caption{max r and zeta}
%\end{figure}
%
%\begin{figure}[H]
%\label{fig:ratio}
%    \includegraphics[width=0.47\textwidth]{pics/d2t}
%    \caption{thing}
%\end{figure}
%
%
%
%
%
%\newpage
%\section{Extra Calculations}
%Using 1mm as a typical hit distance from the wire, we calculate,
%\begin{equation}
%\zeta \approx 0.1943 \approx 0.2
%\end{equation}
%we then get,
%\begin{equation}
%              \tan \Phi \equiv  \frac{u_y}{u_x} = \frac{0.04 \cos \phi \sin \phi}{1 + 0.04 \cos^2 \phi}            
%\end{equation}
%which is maximized at $\pi/4$ with a value of about 0.02. This means the drift in the y direction is less than 2\% of the radial motion. Say we have a charge starting at 2.5 mm, by the time it has reached 0.5mm from the wire, it has moved 0.04mm in y, which is only a 4 degree shift in phi.
%
%
%\newpage
%
%The exact solution for the change in $\phi$ throughout the particle's trajectory is obtained by solving,
%\begin{equation}
%\frac {d\phi}{dx} = \frac{u_y}{x u_x}  = \frac{\zeta^2 \cos \phi \sin \phi}{x(1 + \zeta^2 \cos^2 \phi)}     
%\end{equation}
%which is a first-order nonlinear ODE with a nasty solution, call it $\phi(x,\phi_0)$. Throwing out the third order in $x$ term give,
%\begin{equation}
%d\phi   = C^2  x \cos \phi \sin \phi dx    
%\end{equation}
%or,
%\begin{equation}
% \int_{\phi_i}^{\phi_f} \frac{d\phi}{\cos \phi \sin \phi}   =   \int_{d}^{0}C^2  x  dx    
%\end{equation}
%or,
%\begin{equation}
% \ln(\frac{\tan\phi_f}{\tan\phi_i})  =  \frac{-C^2 x^2}{2}
%\end{equation}
%or,
%\begin{equation}
% \tan\phi_f  = \tan\phi_i \text{e}^\frac{-C^2 x^2}{2}
%\end{equation}
% or,
%\begin{equation}
%\Delta \tan\phi =  \tan\phi_f -\tan\phi_i = \tan\phi_i (\text{e}^\frac{-C^2 x^2}{2} - 1)
%\end{equation} 
% or,
%\begin{equation}
% \phi_f  = \arctan(\tan\phi_i \text{e}^\frac{-C^2 x^2}{2})
%\end{equation} 
% or,
%\begin{equation}
%\Delta \phi = \phi_f - \phi_i  = \arctan(\tan\phi_i \text{e}^\frac{-C^2 x^2}{2}) - \phi_i
%\end{equation}  
%which has a magnitude that is maximized when the exponent is maximized.
%so,
%\begin{equation}
% \frac{\partial \Delta \phi}{\partial \phi_i} = \frac{k \sec^2\phi_i}{k^2 \tan^2\phi + 1} - 1
%\end{equation}  
% for some constant $k$
% man something is fucked up
% 
%
% 
% 
% 
% \newpage
%
%OTHER THINGS
%
%
%\begin{equation}
%t = \int_{a}^{d} \frac{dr}{u_r} 
%\end{equation}
%\begin{equation}
%u_r = \sqrt{u^2_x + u^2_y} = \eta \sqrt{1 + 2\zeta^2 \cos^2 \phi + \zeta^4 \cos^2 \phi}
%\end{equation}
%\begin{equation}
%t = \int_{a}^{d} \frac{dr}{\eta \sqrt{1 + 2\zeta^2 \cos^2 \phi + \zeta^4 \cos^2 \phi}} 
%\end{equation}
%\begin{equation}
% \eta \equiv  \mu |\vec{E}|/(1 + \zeta^2)
%\end{equation}
%\begin{equation}
%t = \frac{1}{v_d} \int_{a}^{d} \frac{(1 + \zeta^2) dr}{ \sqrt{1 + 2\zeta^2 \cos^2 \phi + \zeta^4 \cos^2 \phi}} 
%\end{equation}
%A very bad integral, but the particle WONT spiral! assuming the steady state, it will get locked after a quarter of a turn!!!!
%
%
%
%
%
%The d-to-t function is is then obtained by substituting the form of $\phi(x,\phi_0)$ into the integral,
%\begin{equation}
%t = \int_{a}^{d} \frac{ dx (1 + \zeta^2)}{\mu |\vec{E}| (1 + \zeta^2 \cos^2 \phi(x,\phi_0))} 
%\end{equation}
%
%
%
%
%
%
%\newpage
%\section{The Great Failure}
%Taking only the first two terms in the taylor expansion of $\arctan(d C \cos \phi )$ about $d=0$, we obtain,
%\begin{equation}
%t \approx \frac{d}{v_d}(\frac{1}{\cos\phi} + \frac{d^2 C^2}{3}(1 - \cos \phi))
%\end{equation}
%For now, we will approximate even further to obtain the simple expression
%\begin{equation}
%t \approx \frac{d}{v_d \cos\phi}
%\end{equation}
%which is easily invertible,
%\begin{equation}
%d \approx t v_d \cos\phi
%\end{equation}
%and  we want to find the function $g(t)$ such that,
%
%\
%
%\newpage
%\section{More old}
% naively, we would choose the velocity $\vec{u}(d)$, yielding
%\begin{equation}
%t(d,\phi) = \frac{d}{u_x (d,\phi)} = \frac{d}{v_d}\bigg(\frac{1 + C^2 d^2}{1 + C^2 d^2 \cos^2 \phi}\bigg)
%\end{equation}
%which we call the \emph{first approximation}. 
%A better approximation can be made by assuming that the velocity decreases linearly with $r$. Under this assumption we get,
%\begin{equation}
%r_{\text{avg}}  = \frac{d}{2}
%\end{equation}
%which gives,
%\begin{equation}
%t(d,\phi) = \frac{d}{v_d}\bigg(\frac{1 + (\frac{C d}{2})^2}{1 + (\frac{C d  \cos \phi}{2})^2}\bigg)
%\end{equation}
%We call this the \emph{second approximation}. 
%
%
%
%




